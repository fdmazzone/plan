\documentclass[11pt]{beamer}
\usetheme{Darmstadt}
\usepackage[utf8]{inputenc}
\usepackage[spanish]{babel}
\usepackage{amsmath}
\usepackage{amsfonts}
\usepackage{ltablex}
\usepackage{tabularx}
\usepackage{xtab}
\usepackage{colortbl}
\usepackage{amssymb}
\author{CCP de la Lic. en Matemática}
\title{Reunión Informativa Plan Lic. en Matemática 2022}
%\setbeamercovered{transparent} 
%\setbeamertemplate{navigation symbols}{} 
%\logo{} 
%\institute{} 
\date{16 de mayo de 2022} 
%\subject{} 
\begin{document}

\begin{frame}
\titlepage
\end{frame}

%\begin{frame}
%\tableofcontents
%\end{frame}

\section{Objetivos Reunión}
\begin{frame}{Objetivo de esta reunión}
\begin{block}{}
\begin{itemize}
 
\item<+-> Presentar ante la comunidad departamental el proyecto de nuevo plan
\item<+->  Abrir un canal de recepción de comentarios y sugerencias. 
 \end{itemize}

\end{block}


\end{frame}


\section{Marco institucional}
 
\begin{frame}{Lineamientos curriculares de la UNRC }
\begin{small}

Resoluciones CS 297/17, 298/17 y 008/21
 \onslide<+->
\begin{itemize}

 
\item<+-> Realizar una evaluación del plan de estudio vigente con carácter participativo, involucrando a \textit{todos los actores}.
 \item<+-> Identificar sus alcances actuales,  dificultades y vacancias.
 \item<+-> Contextualización.   Plan inserto en una época, en una región, en una población de estudiantil, etc.
 \item<+-> \textit{Organización curricular mixta.} Espacios de integración interdisciplinaria en torno a problemas de la práctica profesional.
 \item<+-> \textit{Flexibilidad curricular.} En cuanto a correlatividades, materias optativas, electivas
 \item<+-> \textit{Organización de la formación en ciclos}  que nuclean competencias referidas a un tipo de formación.


\end{itemize}
\end{small}
\end{frame}


\section{PIIMEI}

\begin{frame}{Actividades realizadas dentro PIIMEI}
\begin{block}{}
\begin{itemize}
\item<+-> Cursos de capacitación.
\item<+-> Consultas a egresados/as de la carrera y a docentes de nuestro departamento.
\item<+-> Se analizaron los planes de estudios de las licenciaturas en matemática del país
\item<+-> Se hicieron propuestas pedagógicas innovadoras en el marco de las materias del ciclo básico.
\item<+-> Se consultó documentos elaborados por expertos sobre perfiles de las carreras de matemática y sobre el rol del matemático en ámbitos no académicos
\end{itemize}
\end{block}


\end{frame}

\section{Características}


\begin{frame}{Características del nuevo plan}

\begin{itemize}
\item<+-> Carrera de 5 años.
\item<+->Identificar competencias para el Lic. en Matemática.  [Paniagua et al., 2013]. 
\item<+-> Roles del Lic. en ámbitos fuera de la academia. (SIAM) 
  \item<+-> Organización curricular por ciclos y por trayectos.
 \item<+-> Temáticas emergentes (ciencia datos).
\item<+-> Reconocimiento de actividades extracurriculares.
 \end{itemize}


\end{frame}




\section{Ciclos y trayectos}



\begin{frame}{Ciclo Básico}
 \fontsize{10pt}{10pt}\selectfont
\textbf{Ciclo Básico} El ciclo básico incluye 14 asignaturas (comunes con Prof).

\setlength\arrayrulewidth{1pt}
\begin{center}
\begin{tabularx}{1\textwidth}{|>{\raggedleft\arraybackslash}X |
>{\raggedleft\arraybackslash}X |
>{\raggedleft\arraybackslash}X |
>{\raggedleft\arraybackslash}X |}
\hline
  \rowcolor[gray]{.9}
  \emph{Asignaturas del ciclo básico} & Cód.  & \emph{Hs. sem.} &  \emph{Hs. tot.}
  \\\hline
  Introducción al Cálculo                      &   COD1    &           8   &   112          \\ \hline
  Fundamentos del Álgebra    	     &   1904   &           8   &   112          \\ \hline
  Geometría I               	     &   1935   &           6   &    84          \\ \hline
  Análisis Matemático I &   COD2   &           8   &   112          \\ \hline
  
 Álgebra Lineal I                   &   1933   &           8   &   112          \\ \hline

    Geometría Analítica                   &   COD3   &       6       &   84          \\ \hline
 
 Análisis Matemático II &   COD4   &           8   &   112          \\ \hline
  
  Taller de informática              &   1927   &           6   &    84          \\ \hline
  
  Inglés (A)     & 1976 &  4   & 112 \\ \hline
  Probabilidades                     &   1987   &           8   &   112          \\ \hline
  Estructuras Algebraicas            &   1993   &           8   &   112          \\ \hline
   Estadística                        &   1991   &           6   &    84          \\ \hline
  Física                             &   1930   &           6   &    84          \\ \hline
  Sociología de la Educación & 2064 & 4 & 56 \\ \hline
  \emph{Total de horas ciclo básico }& \multicolumn{3}{r|} {\emph{1372}}            \\ \hline
\end{tabularx}
\end{center}



\end{frame}






\begin{frame}{Ciclo Superior}
 \fontsize{11pt}{11pt}\selectfont
\textbf{Ciclo Superior} Consta de 12 materias.
obligatorias.

\begin{center}
\begin{tabularx}{1\textwidth}{|p{5.5cm}|p{1.3cm}|p{1.3cm}|p{1.3cm}|}
\hline
  \rowcolor[gray]{.9}
Asignaturas   & Cód. & Hs. sem.&  Hs. tot.\\ \hline

Cálculo Numérico  Computacional          & 2030   &           8        &     112           \\ \hline
Topología                                & 1917   &           8        &     112           \\ \hline
Álgebra Lineal Aplicada                  & 2261   &           8        &     112           \\ \hline
Fundamentos de Análisis           &   COD5 &             8        &             112    \\ \hline
Variables Complejas   & 1911   &           8        &     112           \\ \hline
Medida e Integración                     & 2263   &           10        &     140           \\ \hline
Ecuaciones Diferenciales                 & 1913   &           8        &     112           \\ \hline
Geometría Diferencial                    & 1915   &           8        &     112           \\ \hline
Modelos de regresión y metodos empíricos  & COD6  &           8
& 112  \\ \hline
Introducción a las Ecuaciones en Derivadas Parciales &   2212  & 8 & 112  \\ \hline
Análisis Funcional               &    1916   &       8        &            112      \\ \hline
\emph{Total de horas ciclo superior }    &\multicolumn{3}{r|}{\emph {1260}   }               \\ \hline

\end{tabularx}
\end{center}
\end{frame}


\begin{frame}{Ciclo Especialización}
 \fontsize{12pt}{12pt}\selectfont
 \begin{center}
\begin{tabularx}{1\textwidth}{|p{5cm}|p{1.5cm}|p{1.5cm}|p{1.5cm}|}
\hline
  \rowcolor[gray]{.9}
\emph{Asignaturas  } & Código&\emph{Horas semanales} & 
\emph{Horas totales}      \\ \hline
Optativa I                     &           &            8        &         112        \\ \hline
 Electiva                 &            &    6                &        84          \\ \hline
 Optativa II                    &            &           10         &          140        \\ \hline
 Trabajo Final                     & 2265         &       10             &          140        \\ \hline
\emph{Total de horas }    & \multicolumn{3}{r|}{\emph{476 }} \\ \hline
\end{tabularx}
\end{center}
\end{frame}
 
 \begin{frame}{Otras Actividades}
 El/la estudiante debe acreditar 60hs en cursos y seminarios extracurriculares y otras actividades. Pudiendo
acreditar
\begin{itemize}
\item<+->Participación en seminarios, por ejemplo el seminario académico que se lleva regularmente adelante
en el Dpto de Matemática.
\item<+->Prácticas Socio-Comunitarias
\item<+->Idiomas
\item<+->Cursos extracurriculares en el marco de congresos.
\item<+->Actividades de Investigación-extensión reconocidas.
\item<+->Participación acreditada en órganos colegiados y centros de estudiantes.
\end{itemize}
 \end{frame}
\begin{frame}{Trayectos}
 \begin{itemize}
  \item<+->\textbf{Trayecto en Matemática Pura (TMP)}
  \item<+->\textbf{Trayecto en Matemática Aplicada (TMA)}
  \item<+->\textbf{Trayecto Ciencia de Datos (TCD).}
  \item<+->\textbf{Trayecto en Tecnologías de la Información (TTI)}
  \item<+->\textbf{Trayecto de Formación Socio-Política-Cultural y Pedagógica (TFSPCP)}
 \end{itemize}

\end{frame}



\section{Listado asignaturas}

\begin{frame}{Listado total de asignaturas}
 \fontsize{9pt}{9pt}\selectfont  
\begin{center}

\begin{tabularx}{1\textwidth}{|>{\raggedleft\arraybackslash}X |
>{\raggedleft\arraybackslash}X |
>{\raggedright\arraybackslash} p{5cm}|
>{\raggedleft\arraybackslash}X |
>{\raggedleft\arraybackslash}X |}
\hline
\hline
\rowcolor[gray]{.9}\multicolumn{5}{|c|}{\textbf{Primer año}}                                                        \\ \hline

\emph{Cuat. }  &\emph{Cód.}  & \emph{Materia}                              &    hr. sem.       &  hr. Tot.        \\ \hline

I              & COD1\footnote{Código identificador para este documento hasta tanto se le asigne un código desde la FCEFQyN}           & Introducción al Cálculo                               &          8  &   112     \\ \hline
I              &  1904        &  Fundamentos de Álgebra                      &          8  &   112       \\ \hline
I              & 1935          & Geometría I                              &          6  &    84       \\ \hline
\multicolumn{3}{|l|}{\textbf{Total de Horas cuatrimestre I}}              &\textbf{22 } &\textbf{308 }\\ \hline
II             & COD2          & Análisis Matemático I                              &          8  &   112       \\ \hline
II             & 1933          & Álgebra Lineal I                         &          8  &   112       \\ \hline
II             & COD3         & Geometría Analítica                    &          6  &    84       \\ \hline

\multicolumn{3}{|l|}{\textbf{Total de Horas cuatrimestre II}}             &\textbf{22 } &\textbf{308 }\\ \hline
 
\rowcolor[gray]{.9}
\multicolumn{5}{|c|}{\textbf{Segundo año}}                                                                \\ \hline
III            & COD4         & Análisis Matemático II                           &          8  &   112      \\ \hline
III             & 1927          & Taller de informática                    &          6  &    84       \\ \hline
III            & 1976          & Inglés (Anual)                           &          4  &    56     \\ \hline
III             & 2261          & Álgebra Lineal Aplicada                  &          8  &   112       \\ \hline


\multicolumn{3}{|l|}{\textbf{Total de Horas cuatrimestre III}}            & \textbf{26 }&\textbf{364 }\\ \hline
IV             & 1987          & Probabilidades                           &          8  &   112      \\ \hline
IV             & 1993          & Estructuras Algebraicas                  &          8  &   112      \\ \hline
IV    &  1976 & Inglés (Anual) & 4  & 56 \\ \hline
\multicolumn{3}{|l|}{\textbf{Total de Horas cuatrimestre IV}}             &\textbf{20 } &\textbf{280 }\\ \hline
\end{tabularx}
\end{center}
\end{frame}

\begin{frame}{Listado total de asignaturas}
 \fontsize{9pt}{9pt}\selectfont  
\begin{center}

\begin{tabularx}{1\textwidth}{|>{\raggedleft\arraybackslash}X |
>{\raggedleft\arraybackslash}X |
>{\raggedright\arraybackslash} p{5cm}|
>{\raggedleft\arraybackslash}X |
>{\raggedleft\arraybackslash}X |}
\hline
\hline
\rowcolor[gray]{.9}\multicolumn{5}{|c|}{\textbf{Tercer año}}                                                        \\ \hline
\emph{Cuat. }  &\emph{Cód.}  & \emph{Materia}                              &    hr. sem.       &  hr. Tot.        \\ \hline
V              & 1991          & Estadística                                &           6         &         84         \\ \hline
V              & COD5         & Fundamentos de Análisis           &            8        &             112     \\ \hline


V            & 2030          & Cálculo Numérico  Computacional          &          8  &   112    \\ \hline

\multicolumn{3}{|l|}{\textbf{Total de Horas cuatrimestre V}}                &\textbf{22}           &\textbf{308}         \\ \hline
VI             & 1930          & Física                                     &            6        &           84       \\ \hline
VI             & 1911          & Variables Complejas                          &             8       &         112         \\ \hline
VI            & 1917          & Topología                                   &              8      &        112          \\ \hline
\multicolumn{3}{|l|}{\textbf{Total de Horas cuatrimestre VI} }              &\textbf{22}           &\textbf{308}         \\ \hline
  
\rowcolor[gray]{.9}\multicolumn{5}{|c|}{\textbf{Cuarto año}}                                                        \\ \hline
VII           & 2263          & Medida e Integración                       &             10       &            140     \\ \hline
VII            & 1913          & Ecuaciones Diferenciales                   &           8         &        112          \\ \hline
VII           & 2064          & Sociología de la Educación                           &          4  &    56     \\ \hline
\multicolumn{3}{|l|}{\textbf{Total de Horas cuatrimestre VII}}              & \textbf{22}          &\textbf{308}         \\ \hline
VIII             & 1915          & Geometría Diferencial                      &              8     &        112          \\ \hline
VIII           & COD6           & Modelos de regresión y metodos empíricos                  &             8     &         112         \\ \hline

% VIII           & 2265          & Modelos Matemáticos                        &               6     &       84           \\ \hline

VIII          & 2212           & Introducción a las Ecuaciones en Derivadas Parciales & 8  & 112  \\ \hline

\multicolumn{3}{|l|}{\textbf{Total de Horas cuatrimestre VIII}}             & \textbf{24}          &\textbf{336}         \\ \hline
\end{tabularx}
\end{center}
\end{frame}


\begin{frame}{Listado total de asignaturas}
 \fontsize{9pt}{9pt}\selectfont  
\begin{center}

\begin{tabularx}{1\textwidth}{|>{\raggedleft\arraybackslash}X |
>{\raggedleft\arraybackslash}X |
>{\raggedright\arraybackslash} p{5cm}|
>{\raggedleft\arraybackslash}X |
>{\raggedleft\arraybackslash}X |}
\hline
\hline
\rowcolor[gray]{.9}\multicolumn{5}{|c|}{\textbf{Quinto año}}                                                        \\ \hline
\emph{Cuat. }  &\emph{Cód.}  & \emph{Materia}                              &    hr. sem.       &  hr. Tot.        \\ \hline
IX            &  1916         &  Análisis Funcional               &          8          &            112      \\ \hline
IX           &               & Optativa I                                &            10      &         140        \\ \hline
X           &           & Electiva                             &    6                &        84          \\ \hline
\multicolumn{3}{|l|}{\textbf{Total de Horas cuatrimestre VII}}              & \textbf{22}          &\textbf{336}         \\ \hline
X           &               & Optativa II                                &           10         &          140        \\ \hline
X           & 2265          & Trabajo Final                              &       10             &          140        \\ \hline


\multicolumn{3}{|l|}{\textbf{Total de Horas cuatrimestre VIII}}             & \textbf{20}          &\textbf{280}         \\ \hline
\multicolumn{4}{|l|}{\textbf{Total de Horas del Plan de estudios}}                               &\textbf{3136}         \\ \hline
\end{tabularx}
\end{center}
\end{frame}




\section{Roles CCP}

\begin{frame}{Roles CCP}

\begin{itemize}
 \item<+-> \textbf{Coordinar y monitorear la ejecución del plan}
 \item<+->\textbf{Propiciar encuentros, seminarios, cursos prefeccionamiento docente}
\end{itemize}






\end{frame}

\end{document}
