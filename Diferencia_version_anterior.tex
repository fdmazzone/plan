---------------------- PlanLicMatematica2022.tex --------------------------
index 163f271..aa7e9eb 100644
@@ -18,9 +18,9 @@
 %\setsansfont{Gentium Basic}
 \usepackage{colortbl}
 \usepackage{graphics}
-
+\usepackage{ltablex}
 \usepackage{cite}
-
+\usepackage{lscape}
 
 
 \newenvironment{colortext}[1]{\color{#1}}{\ignorespacesafterend}
@@ -95,11 +95,7 @@ Plan de Estudios de la
 Carrera de Licenciatura en Matemática, de la Facultad de Ciencias Exactas, 
 Físico - Químicas y Naturales, de la Universidad Nacional de Río Cuarto. Que reemplaza el Plan de Estudio de la Licenciatrura en Matemática aprobado por resolución del Consejo Directivo Nº 156/08, 
 ratificada por resolución del Consejo Superior Nº 212/08.
-% Resolución del Consejo Directivo Nº 258 /07, ratificada 
-%por Resolución del Consejo Superior Nº 289/07. Registro y toma de conocimiento 
-%por parte de la Dirección Nacional de Gestión Universitaria informado por nota 
-%Nº 544/08. Introducción de modificaciones, que generaron la versión 1 del Plan 
-%de Estudios, aprobadas por 
+
 
 \section{Responsables del proyecto}
 
@@ -110,7 +106,7 @@ la Facultad de Ciencias Exactas Físico-Químicas y Naturales.
 
 \subsection{Unidad académica responsable de la implementación del proyecto}
 
-Facultad de Ciencias Exactas Físico-Químicas y Naturales.
+Facultad de Ciencias Exactas Físico-Químicas y Naturales. 
 
 
 
@@ -118,39 +114,48 @@ Facultad de Ciencias Exactas Físico-Químicas y Naturales.
 
 \subsection{Razones que justifican la creación y o los cambios curriculares del proyecto de formación  y que justifican su realización}
 
-A continuación se enumeran normativas que modificaron las existentes al momento de crear la versión anterior del plan de estudios como así también instancias de intercambio institucional que delinearon nuevos paradigmas en la elaboración de planes de estudios e introdujeron nuevas variables a tener en cuenta.
+La actualización  contenida en esta propuesta da cuenta de cambios en las normativas de la UNRC en cuanto a planes de estudio y a lineamientos para la elaboración de los  mismos. Más específicamente se trató de ajustar el Plan de Estudios de la Lic. en Matemática a:
+
+
+
 \begin{enumerate}
- \item Que por Resolución CS-UNRC 297/2017 se aprobó el documento ``Hacia   un   currículo contextualizado, flexible e integrado. Lineamientos para la orientación de la innovación  curricular'' que define dimensiones que la Universidad considera importantes a la hora de elaborar planes de estudios, en particular dimensiones epistemólogico-metodológicas, de contextualización, organización, de flexibilidad e integración curricular. 
+ \item La Resolución CS-UNRC 297/2017 que aprobó el documento ``Hacia   un   currículo contextualizado, flexible e integrado. Lineamientos para la orientación de la innovación  curricular'' que define dimensiones que la Universidad considera importantes a la hora de elaborar planes de estudios, en particular dimensiones epistemólogico-metodológicas, de contextualización, organización, de flexibilidad e integración curricular. 
 
- \item Que por Resolución CS-UNRC 298/2017 se implementa el Proyecto de Innovación e Investigación para el Mejoramiento Estratégico Institucional (PIIMEI). Como parte de este Proyecto, la Comisión Curricular Permanente de la Licenciatura en Matemática junto con docentes y alumnos de la carrera emprendió una investigación del currículo de la carrera.  Parte de las conclusiones obtenidas fueron plasmandas en un Informe ``Actividades de Investigación Evaluativa
-Licenciatura en Matemática'' el cual fue evaluado por expertos en curriculo universitario, los cuales sugierieron acciones a seguir.
-\item Que por Resolución CS-UNRC 510/2017 se actualizó el Plan Estratégico Institucional (PEI 2017-2023) el cual se constituye como documento orgánico con miras al desarrollo integral de la universidad, con emplazamiento geográfico y social. Los lineamientos del PEI de la UNRC representan la plataforma desde donde avanzar en la proyección de políticas institucionales de la Universidad en su conjunto y para nuestra Facultad en particular, que atienden necesidades actuales y proponen caminos de actuación a futuro.
-\item Que por Resolución CD-FCEFQyN 410/2019 se aprueba el Plan Estratégico de la Facultad de Ciencias Exactas Físico-Químicas y Naturales (PEExa 2019-2023). En particular, en el Capítulo III, Sección 1 define objetivos de la institución para la enseñanza de grado.   
+ \item La Resolución CS-UNRC 298/2017 que implementa el Proyecto de Innovación e Investigación para el Mejoramiento Estratégico Institucional (PIIMEI). Como parte de este Proyecto, la Comisión Curricular Permanente de la Licenciatura en Matemática junto con docentes y alumnos de la carrera emprendió una investigación del currículo de la carrera.  Parte de las conclusiones obtenidas fueron plasmandas en un Informe ``Actividades de Investigación Evaluativa
+Licenciatura en Matemática'' el cual fue evaluado por expertos en curriculo universitario.
 
- \item Que por Resolución CS-UNRC 008/2021 se establecen los conceptos, normas y procedimientos que regulan los procesos de elaboración, presentación, formalización, aprobación, seguimiento, evaluación y tramitación de reconocimiento de Nuevos Planes de Estudio y de modificaciones que impliquen nuevas versiones de los Planes de Estudio existentes.
+\item La Resolución CS-UNRC 510/2017 se actualizó el Plan Estratégico Institucional (PEI 2017-2023) el cual se constituye como documento orgánico con miras al desarrollo integral de la universidad, con emplazamiento geográfico y social. Los lineamientos del PEI de la UNRC representan la plataforma desde donde avanzar en la proyección de políticas institucionales de la Universidad en su conjunto y para nuestra Facultad en particular, que atienden necesidades actuales y proponen caminos de actuación a futuro.
+
+\item La Resolución CD-FCEFQyN 410/2019 que aprueba el Plan Estratégico de la Facultad de Ciencias Exactas Físico-Químicas y Naturales (PEExa 2019-2023). En particular, en el Capítulo III, Sección 1 define objetivos de la institución para la enseñanza de grado.   
+
+ \item La Resolución CS-UNRC 008/2021 que establece los conceptos, normas y procedimientos que regulan los procesos de elaboración, presentación, formalización, aprobación, seguimiento, evaluación y tramitación de reconocimiento de Nuevos Planes de Estudio y de modificaciones que impliquen nuevas versiones de los Planes de Estudio existentes.
 
 \end{enumerate}
 
 \subsection{Razones que determinan la conveniencia de la implementación de proyecto curricular  y que justifican su realización.}
 
-Implementar los lineamientos propuestos por la Facultad dentro del proyecto PIIMEI. Entre ellos se enumeran:
+La presente propuesta incorpora o profundiza los siguientes aspectos en la elaboración de planes de estudio: 
+
+\begin{enumerate}
+\item Contextualización y visión totalizante. Se analizaron las características de la población concreta de estudiantes a los que va dirigida la propuesta, las caracteríticas en cuanto a formación e intereses de investigación del plantel de docentes que ejecutará el plan, nuevas áreas emergentes, como el caso del análisis de grandes conjuntos de datos. Además se tomó en consideración aspectos administrativos, como por ejemplo aquellos que devienen del uso de la plataforma SIAL. 
+\item Flexibilidad curricular. Al ciclo de optativas y trabajo final se agrega una materia electiva.
+\item Organización curricular mixta. Se definen ciclos de formación y problemas transversales en la práctica docente. 
+\item Transversalidad de la práctica profesional. Este punto es consecuencia del anterior.
+\item Incorporación de espacios de formación socio-política-cultural con la incorporación de xxxxxx
+\end{enumerate}
+
+A esto se suma la necesidad de resolver las siguientes cuestiones más coyunturales del plan vigente.
 
 \begin{enumerate}
-\item Contextualización del currículo y visión totalizante.
-\item Flexibilidad curricular.
-\item Organización curricular mixta.
-\item Transversalidad de la práctica profesional.
-\item Formación en ciclos o trayectos.
-\item Práctica docente de manera transversal a todo el plan de estudio.
-\item Inconsistencia de la carga horaria de asignaturas existentes.
-\item Actualización curricular.
-\item Definición con mayor flexibilidad del ciclo de especialización.
-\item Incorporación de espacios de formación socio-cultural.
+\item Inconsistencia de la carga horaria de asignaturas homólogas en distintos planes de estudios (Probabilidades 1987).
 
+\item Se modifican correlatividades atendiendo a necesidades emergentes de nuevos desafíos en la enseñanza (Probabilidades 1987). 
+
+\item Mayor flexibilidad del ciclo de especialización. No se especifican materias optativas, en cuanto a contenidos ni siquiera nombres. Se detectó que en los planes previos la existencia de diversas materias optativas con diferentes denominaciones causaba problemas en cuanto a la ejecución del plan a nivel administrativo.  
 \end{enumerate}
 
-\subsection{Correspondencia con los fines y objetivos de la
-Universidad} 
+
+\subsection{Correspondencia con los fines y objetivos de la Universidad}
 
 Los fines y objetivos de la Universidad y de la Facultad de Ciencias, Exáctas Físico-Químicas y Naturales están definidos en el Estatuto Universitario y en el Plan Estratégico Institucional (PEI) y en el Plan Estratégico  de la Facultad (PEExa). El plan de estudios de la Licenciatura de Matemática se enmarca dentro de los objetivos y fines declarados en los anteriores documentos, especialmente por las consideraciones en ellos establecidas  que enumeramos debajo. 
 
@@ -221,51 +226,20 @@ En ella se consignan las áreas y temas de interés, en particular las área 8 d
 
 \subsubsection{Breve reseña del origen y trayectoria de la carrera, considerando los ámbitos nacional, regional e institucional.}
 
-Es dificil rastrear los orígines de la carrera de Licenciatura en Matemática en Argentina. La enseñanza de esta ciencia en territorio argentino se remonta a la  época colonial. Instaurado el primer gobierno patrio, como es sabido, Manuel Belgrano fue un impulsor del estudio de las ciencias. Citando a Edgardo L. Fernández Stacco
- (ver \cite{stacco2011200})
- : ``Producida la revolución, Manuel Belgrano, vocal de la Junta de Gobierno, hizo
-crear una Escuela de Matemáticas que puso bajo la dirección del teniente coronel
-catalán Felipe de Sentenach, instalada en el 12 de setiembre de 1810. Tenía como
-función formar a los oficiales del ejército.
-La enseñanza comprendía: aritmética, geometría plana, trigonometría rectilínea
-y geometría práctica. Además los oficiales aspirantes a la ingeniería y artillería debían
-cursar: Álgebra inferior y superior con sus aplicaciones a la aritmética y la geometría;
-mecánica y en particular, estática; secciones cónicas.''
+ La enseñanza de la matemática en territorio argentino se remonta a la  época colonial. Instaurado el primer gobierno patrio,  Manuel Belgrano fue un impulsor del estudio de las ciencias con  la creación de la   Escuela de Matemáticas  el 12 de setiembre de 1810 (ver \cite{stacco2011200}). 
 
  Una de las primeras menciones que hemos hallado de una carrera denominada Licenciatura en Matemática es en  \cite{ortiz2011julio} 
- donde se menciona que en  1926 se creo una Licenciatura y un Doctorado en Matemática y en Física dentro de una Facultad de Ingeniería. ``Aquella carrera incluía cursos avanzados de matemática pura (análisis y geometría superior) y dos cursos regulares de física-matemática en reemplazo del antiguo curso único de física-matemática del último año de los planes anteriores.Todas estas realizaciones son claramente indicativas de que se comenzaba a prestar mayor  atención a los aspectos teóricos de la ciencia.''
-
- El estudio de la matemática se ha diseminado por todo el sistema de educación superior Argentino, en la actualidad la carrera de Licenciatura en Matemática se ofrece en las siguientes universidades nacionales: UNRC, UNAB, UNSL, UNC, UNLPam, UNNE, UNICEN, UNL, UNCOMA, UNMDP, UNSJB, UNS, UNSE, UNR y UBA.  
- 
- En la UNRC el plan de estudios original data de 1975. Fue elaborado por xxxxxxxx. Posteriormente el plan fue modificado en 1993, 2001 y 2008.  
- 
- 
- 
-
-\begin{description}
- \item[Plan 1975] La carrera de Lic. en Matemática fue una carrera de 5 años de duración que dió inicio en el año 1975 y tuvo su reconocimiento oficial en la Resolución Ministerial N° 1560/80. 
+ donde se menciona que en  1926 se creo una Licenciatura y un Doctorado en Matemática y en Física dentro de una Facultad de Ingeniería. 
 
- \includegraphics[scale=.3]{plan1975.png}
+ Posteriormente el estudio de la matemática superior se ha diseminado por todo el sistema de educación superior Argentino, siendo una de las carreras con más larga trayectoria en el país. En la actualidad la carrera de Licenciatura en Matemática se ofrece en las siguientes universidades nacionales: UNRC, UNAB, UNSL, UNC, UNLPam, UNNE, UNICEN, UNL, UNCOMA, UNMDP, UNSJB, UNS, UNSE, UNR y UBA.  
  
-\item[Plan 1993] Carrera de 5 años
-
- \includegraphics[scale=.3]{plan1993.png}
-
-\item[Plan 2001] Carrera de 5 años
-
- \includegraphics[scale=.5]{plan2001.png}
-
-
-\item[Plan 2008] Carrera de 4 años. Resolución del CD-FCEFQyN Nº 258 /07, ratificada 
+ En la UNRC la carrera de Lic. en Matemática fue creada desde el origen de la universidad. Su primer plan de estudios fue aprobado por Resolución Ministerial  N° 1560/80 en el año 1975. Consistió en una carrera de 5 años de duración Posteriormente el plan fue modificado en 1993, 2001 y 2008. El plan de 1993 incorporó como una de sus características centrales la elaboración de un trabajo final. Además se introdujeron cambios en pos de articular con las carreras de computación recientemente creadas.  En el plan de 2001 se revierten en parte los cambios anteriores, separando algunas  materias respecto a las correspondientes de las carreras de  computación y se introducen cambios en las áreas de geometría y estadística.  
+ El plan de  2008 llevó a la carrera a 4 años. Entre los aspectos centrales contenidos en este plan citamos que se hizo explícito el objetivo de incorporar la formación interdisciplinaria. La implementación del mismo obedeció en parte a recomendaciones elaboradas por la Unión Matemática Argentina en su documento \cite{uma}.  Fue aprobado por Resolución del CD-FCEFQyN Nº 258 /07, ratificada 
 por Resolución del CS-UNRC Nº 289/07. Registro y toma de conocimiento 
 por parte de la Dirección Nacional de Gestión Universitaria informado por nota 
 Nº 544/08. Introducción de modificaciones, que generaron la versión 1 del Plan 
 de Estudios, aprobadas por resolución del CD-FCEFQyN Nº 156/08, 
-ratificada por resolución del CS-UNRC Nº 212/08.  Nueva introducción de modificaciones, que generaron la versión 2 aprobada por Resolución del CD-FCEFQyN N° 340/17, por la cual se aprueba el Texto Ordenado del Plan de Estudios 2008, Versión 2, de la Carrera de "Licenciatura en Matemática", según consta en el Anexo 1 de la citada Resolución, obrante a fojas 180/216 del Expediente N° 88664.  Resolución del CS-UNRC N° 443/17, por la cual se ratifica la Resolución  CD-FCEFQyN N° 340/17.  La versión 2 del Plan 2008 aún no se encuentra vigente,
-
- \includegraphics[scale=.75]{plan2008.png}
-
-\end{description}
+ratificada por resolución del CS-UNRC Nº 212/08.  Nueva introducción de modificaciones, que generaron la versión 2 aprobada por Resolución del CD-FCEFQyN N° 340/17, por la cual se aprueba el Texto Ordenado del Plan de Estudios 2008, Versión 2, de la Carrera de "Licenciatura en Matemática", según consta en el Anexo 1 de la citada Resolución, obrante a fojas 180/216 del Expediente N° 88664.  Resolución del CS-UNRC N° 443/17, por la cual se ratifica la Resolución  CD-FCEFQyN N° 340/17.  La versión 2 del Plan 2008 aún no se encuentra vigente. 
 
 
 
@@ -275,7 +249,9 @@ ratificada por resolución del CS-UNRC Nº 212/08.  Nueva introducción de modif
 \begin{description}
  \item[Docencia] Como se mencionó hay una larga trayectoria  de nuestra Facultad en el dictado de carreras de grado y posgrado  vinculadas con la matemática, particularmente la Licenciatura en Matemática, el Profesorado en Matemática, Maestría en Matemática Aplicada y Especialidad en Didáctica de la Matemática. Además de estos antecedentes, merece mencionarse que diversas carreras de nuestra y de otras facultades requieren el dictado de materias vinculadas con la matemática y por tanto es necesario la formación de docentes altamente calificados en esta disciplina. Nuestros egresados forman parte de los departamentos de matemática de otras facultades de nuestra universidad.
  
- \item[Investigación] Dentro del departamento de matemática de la FCEFQyN se ejecutan regularmente  proyectos financiados tanto por SECyT-UNRC como por organismos de financiamiento nacionales (ANPCyT-CONICET). La modelización matemática y la ciencia de datos proveen una metodología esencial a otras áreas del saber, ingenierías, física, biología, geología, economía, etc. Profesionales egresados de nuestro departamento ejecutan tareas de investigación en las áreas mencionadas en nuestra universidad y en otras instituciones.
+ \item[Investigación] Dentro del departamento de matemática de la FCEFQyN se ejecutan regularmente  proyectos financiados tanto por SECyT-UNRC como por organismos de financiamiento de actividades científicas de orden  nacional como ANPCyT y CONICET. Investigadores y becarios del CONICET desarrollan sus actividades en la mencionada unidad académica. Se participa de proyectos y grupos interdisciplinarios, por ejemplo en  facultades de ingeniería, miembros del departamento forman parte  del Instituto de Investigaciones Sociales, Territoriales y  Educativas.   
+ 
+
  
  
 \end{description}
@@ -289,10 +265,10 @@ ratificada por resolución del CS-UNRC Nº 212/08.  Nueva introducción de modif
 
 En la elaboración de este plan de estudios se tuvieran en cuenta las siguientes experiencias y antecedentes.
 \begin{description}
- \item[Unión Matemática Argentina]  La asociación que agrupa a los matemáticos del país ha convocado a matemáticos de reconicida trayectoria a nivel internacional quienes elaboraron un documento (ver \cite{uma}) 
+ \item[Unión Matemática Argentina]  Es la asociación que agrupa a los matemáticos del país.  La UMA convocó a matemáticos de reconicida trayectoria a nivel internacional, quienes elaboraron un documento (ver \cite{uma}) 
  dando cuenta de sugerencias curriculares para la carrera de Licenciatura en Matemática.
  
- \item[Foro UMA-CUCEN] En el marco de las reuniones periódicas del Consejo Universitario Ciencias Exactas y Naturales (CUCEN) se realizó durante los años 2017-2018 un foro donde se debatieron ciclos de formación con el propósito de favorecer la movilidad de los estudiantes entre carreras de Licenciatura en Matemática del país. Nuestra carrera participó activamente de este foro. Es de destacar que fruto de esta participación, y como parte del proyecto PIIMEI, se hizo una comparativa entre los distintos planes, más específicamente se definieron distintas nudos conceptuales y se identificaron las carreras de Licenciatura en Matemática de Argentina que trabajan dichos nudos. \textcolor{red}{NOMBRAR INFROME REALIZADO PIIMEI}
+ \item[Foro UMA-CUCEN] En el marco de las reuniones periódicas del Consejo Universitario Ciencias Exactas y Naturales (CUCEN) se realizó durante los años 2017-2018 un foro donde se debatieron ciclos de formación con el propósito de favorecer la movilidad de los estudiantes entre carreras de Licenciatura en Matemática del país. Nuestra carrera participó activamente de este foro. Es de destacar que fruto de esta participación, y como parte del proyecto PIIMEI, se hizo una comparativa entre los distintos planes, más específicamente se definieron distintas nudos conceptuales y se identificaron las carreras de Licenciatura en Matemática de Argentina que trabajan dichos nudos (ver \cite{CCP}). 
 
 \item[Sistema Nacional de Reconocimiento Académico (SNRA)] Fue creado por la Resolución Ministerial N° 1870/16. Es un sistema voluntario de acuerdos entre instituciones de Educación Superior de la Argentina, que permite el reconocimiento de trayectos formativos (tramos curriculares, ciclos, prácticas, asignaturas, materias u otras experiencias formativas) para que los estudiantes transiten por el sistema aprovechando toda su diversidad y profundizando la experiencia de formación.
 
@@ -310,7 +286,7 @@ Para el diseño de la presente propuesta curricular se tuvo en consideración el
 que estudia perfiles del egresado y escenarios de futuro para el Área de Matemática y la profesión y estrategias de enseñanza, aprendizaje y evaluación de las competencias propias de los profesionales del área. 
 
 \item[Competencias matemáticas para la industria] 
-Es una preocupación permanente en el diseño del plan de estudios de la Licenciatura en Matemática, tanto en esta institución como otras, la inserción del egresado en ámbitos no académicos. Generalmente se refiere a estos ámbitos como la ``industria'' en la bibliografía, aunque se comtempla que estos incuyen organismos públicos, empresas informáticas, etc. Diversas organizaciones se han encargado de identificar aquellas competencias que son requeridas en industrias a profesionales y que pueden ser provistas por egresados del área de las matemáticas y han propuesto estrategias pedagógicas para la consecución de estas competencias. Hemos estudiado los siguientes antecedentes en esta materia.
+Es una preocupación permanente en el diseño del plan de estudios de la Licenciatura en Matemática, tanto en esta institución como otras, la inserción del egresado en ámbitos no académicos. Generalmente  en la bibliografía se refiere a estos ámbitos como la ``industria'',  se contempla que estos incuyen organismos públicos, empresas informáticas, etc. Diversas organizaciones se han encargado de identificar aquellas competencias que son requeridas en industrias a profesionales y que pueden ser provistas por egresados del área de las matemáticas y han propuesto estrategias pedagógicas para la consecución de estas competencias. Hemos estudiado los siguientes antecedentes en esta materia.
 
 La Society for Industrial and Applied Mathematics (SIAM)  es una asociación académica dedicada al uso de las matemáticas en la industria que tiene conexiones con la Asociación Argentina de Matemática Aplicada Computacional e Industrial. La SIAM publicó varios documentos sobre la problemática del matemático en la industria, en particular \cite{society1996siam,society2012siam}.
 
@@ -329,50 +305,28 @@ La población destinataria de la carrera es la definida por la Resolución CS-UN
 
 \subsubsection{Rasgos y características de la población estudiantil que atiende: tener en cuenta los destinatarios reales y potenciales de la formación, considerando las condiciones y cualidades sociales, culturales y económicas que la caracterizan en general y según los grupos de procedencia: diversidades culturales, mayores de 25 años sin título secundario, culturas juveniles vigentes y emergentes, personas en situación de discapacidad, adultos mayores, estudiantes de pueblos originarios, de diferentes lugares del país y extranjeros, entre otros.}
 
-\textcolor{red}{Cómo caracterizaríamos nuestros estudiantes? VER ARCHIVO COMPARTIDO POR MARCELO}
+\textcolor{red}{Cómo caracterizaríamos nuestros estudiantes? SINTETIZAR  ARCHIVO COMPARTIDO POR MARCELO}
 
 \section{Objetivos del proyecto}
-\subsection{Objetivos generales}
-
-Formar egresados con un alto conocimiento técnico en una
-disciplina considerada estratégica en el desarrollo futuro del
-país (ver por ejemplo la convocatoria a subsidios PAV efectuada
-por la SECYT de la Nación \cite{pav2003}).
-
-
-Capacitar para el uso de las herramientas matemáticas en la
-resolución de problemas científicos y/o tecnológicos.
-
-Brindar  al egresado  conocimientos sólidos en la disciplina
-matemática que le permita acceder a carreras de posgrado y/o
-participar en grupos de trabajo interdisciplinario.
-
-Formar profesionales capacitados en la metodología de la investigación matemática y capaces de plantear y resolver problemas en esa área del saber.
 
-\subsection{Objetivos específicos}
 
+Formar un egresado:
 
 \begin{enumerate}
 
-\hyphenation{a-sig-na-tu-ras}
-    \item Profundizar la  formación interdisciplinaria del
-    egresado.
-    \item Incorporar nuevos  recursos
-    informáticos en las distintas asignaturas.
-    \item Fortalecer la integración curricular con el Profesorado
-    en Matemática preservando las identidades de cada carrera.
-    \item Promover un cambio paulatino en las metodologías de
-    enseñanza de la matemática en nuestro departamento.
+ \item Con capacidad crítica y autocrítica,
+ \item Respetuoso de los valores democrácticos y de la diversidad cultural,  
+  \item Con un alto conocimiento técnico en la disciplina,
+ \item Capacitado en la aplicación de la matemática en la
+resolución de problemas científicos y/o tecnológicos,
+\item Que pueda acceder a carreras de posgrado 
+nacionales y del extranjero, 
+\item Capaz de integrarse en grupos de trabajo interdisciplinarios,
+\item Capaz de plantear y resolver problemas de matemática pura,
 
-    \item Alentar al estudiante para que tome contacto con problemas de
-    investigación en distintos momentos de la carrera culminando
-    en un ciclo de especialización con una marcada formación en
-    investigación.
+\end{enumerate}
 
-    \item Brindar al alumno la posibilidad de integrarse tempranamente
-    a las temáticas de su preferencia.
 
-\end{enumerate}
 
 
 
@@ -389,127 +343,130 @@ Formar profesionales capacitados en la metodología de la investigación matemá
 \hyphenation{de-sa-rro-llo}
     \begin{enumerate}
 
-        \item Participar en equipos interdisciplinarios realizando
-         tareas de asesoramiento en temas específicos.
-        \item  Realizar actividades de investigación en proyectos de matemática
-        pura o aplicada.
-       \item Intervenir  como peritos matemáticos en instituciones tales como
-       empresas que realicen desarrollos tecnológicos, bancos,
-       compañías de seguro, etc.
+        \item Participar en equipos interdisciplinarios que utilicen la matemática.
+        
+        \item  Realizar actividades de investigación en proyectos de matemática         pura o aplicada.
+       
+       \item Intervenir  como peritos matemáticos en  organismos públicos o privados  tales como, INDEC,  empresas que realicen desarrollos tecnológicos, bancos,   compañías de seguro, etc.
+       
         \item Acceder a carreras de posgrado.
-         \item Participar de los equipos docentes dirigidos a la
+       
+       \item Participar de los equipos docentes dirigidos a la
          enseñanza de la matemática en los niveles superiores de enseñanza.
+    
+       
+         
     \end{enumerate}
 
 \subsection{Actividades profesionales reservadas al título (Incumbencias)}
     
 \subsection{Perfil del Título}
-\textcolor{blue}{asesorarse}
+ 
 
 
 \subsubsection{Conocimientos que constituyen el fundamento teórico-metodológico de su accionar profesional o académico.}
 
 Se aspira a que el Lic. en Matemática adquiera un conocimiento sólido en las siguientes  áreas de
- la matemática: Análisis Matemático, Funciones de una Variable Compleja,   Teoría de la Medida, Probabilidades y Estadística, Ciencia de Datos, Geometría Diferencial, Álgebra Lineal, Estructuras Algebracicas, Ecuaciones Diferenciales ordinarias y parciales, Cálculo Numérico, Análisis Funcional y Modelización Matemática.
+ la matemática: Análisis Matemático, Funciones de una Variable Compleja,   Teoría de la Medida, Probabilidades. Estadística. Ciencia de Datos, Geometría Diferencial, Álgebra Lineal, Estructuras Algebracicas, Ecuaciones Diferenciales ordinarias y parciales, Cálculo Numérico, Análisis Funcional.
+ 
+ Se pretende además que el estudiante adquiera conceptos básicos de programación, mecánica y en una materia proveniente de otra ciencia, por ejemplo biología, economía, química, física, ingeniería, informática. Para esto último se prevé una  materia electiva. 
 
  Se aspira además a que el estudiante logre una formación complementaria en un área de su elección dentro de la oferta de que disponga como parte de su ciclo de especialización.
  
 \subsubsection{Capacidades y habilidades requeridas para la realización de las actividades que le incumben.}
 
+
+Se espera lograr un profesional capacitado para:
 \begin{enumerate}
 
-\item {Responsabilidad social y compromiso
-ciudadano.} 
+\item { Actuar con responsabilidad social y compromiso
+ciudadano,} 
 
 
 
-\item {Capacidad de aprender, actualizarse y trabajar de manera autónoma.} 
+\item {Aprender, actualizarse y trabajar de manera autónoma,} 
  
 
 
-\item {Capacidad crítica y autocrítica.} 
+\item {Realizar análisis  críticos y autocríticos,} 
  
+\item {Plantear y resolver problemas de matemática pura,} 
 
-\item {Valoración y respeto por la diversidad
-y multiculturalidad.} 
+\item{Idear demostraciones,}
+
+\item {Valorar y respetar  la diversidad
+y la multiculturalidad,} 
  
 
 
 
 
-\item {Dominio de los conceptos básicos
-de la matemática superior.} 
  
 
-\item {Capacidad para construir y desarrollar
+\item {Construir y desarrollar
 argumentaciones lógicas con una
-identificación clara de hipótesis y conclusiones.} 
+identificación clara de hipótesis y conclusiones,} 
  
-\item {Capacidad de abstracción (extraer de una situación los rasgos más relevantes).} 
+\item {La abstracción (extraer de una situación los rasgos más relevantes),} 
  
 
 
-\item {Capacidad para formular problemas
-en lenguaje matemático.} 
+\item {Formular problemas
+en lenguaje matemático,} 
  
 
   
 
 
-\item {Capacidad para iniciar investigaciones
-matemáticas bajo orientación de experto.} 
+\item {Iniciar investigaciones
+matemáticas bajo orientación de experto,} 
  
 
 
-\item {Capacidad para formular problemas
-de optimización, tomar decisiones e interpretar
-las soluciones en contextos originales
-de los problemas.} 
+ 
  
 
 
-\item {Capacidad para contribuir en la
+\item {Contribuir en la
 construcción de modelos matemáticos a
-partir de situaciones reales.} 
+partir de situaciones reales,} 
  
 
 
-\item {Capacidad para utilizar las herramientas
+\item {Utilizar las herramientas
 computacionales de cálculo numérico
 y simbólico para plantear y resolver
-problemas.} 
+problemas,} 
  
 
 
-\item {Capacidad para extraer información
-cualitativa de datos cuantitativos.} 
+\item {Analizar  grandes conjuntos de datos,} 
  
 
 
-\item {Capacidad para expresarse correctamente
+\item {Expresarse correctamente
 utilizando el lenguaje de la
-matemática.} 
+matemática,} 
  
 
 
-\item {Capacidad para comunicarse
-con otros profesionales no matemáticos.} 
+\item {Comunicarse
+con otros profesionales no matemáticos,} 
 
  
 
-\item {Capacidad para actuar en contextos educativos y planificar actividades de enseñanza} 
+\item {Actuar en contextos educativos y planificar actividades de enseñanza,} 
  
 
 
-\item {Conocimiento del inglés para
-leer, escribir y exponer documentos en
+\item {Leer, escribir y exponer documentos en
 inglés, así como comunicarse con otros
-especialistas.} 
+especialistas,} 
 
  
 
 
-\item {Capacidad para trabajar en equipos
+\item {Trabajar en equipos
 interdisciplinarios.} 
 
 \end{enumerate}
@@ -524,114 +481,106 @@ Los fijados por el ``Régimen  de estudiantes y de enseñanza de pregrado y grad
 
 
 \subsection{Organización del Plan de Estudios}
-\begin{colortext}{blue}
+
 \subsubsection{Ciclos, Trayectos  y Áreas} El Plan de Estudios se desarrollará en tres ciclos:
 un ciclo básico común con el profesorado, un ciclo superior y un
 ciclo de especialización.
 
-\paragraph{Ciclo Básico} El ciclo básico incluye 12 asignaturas.
+\paragraph{Ciclo Básico} El ciclo básico incluye 11 asignaturas.
 
 \setlength\arrayrulewidth{1pt}
 \begin{center}
-\begin{tabular}{|l|r|r|r|}\hline
+\begin{tabularx}{1\textwidth}{|>{\raggedleft\arraybackslash}X |
+>{\raggedleft\arraybackslash}X |
+>{\raggedleft\arraybackslash}X |
+>{\raggedleft\arraybackslash}X |}
+\hline
   \rowcolor[gray]{.9}
   \emph{Asignaturas del ciclo básico} & Código  & \emph{Horas semanales} &  \emph{Horas totales}
   \\\hline
-  Cálculo I                          &   1921   &           8 hs  &   120 hs         \\ \hline
-  Matemática Discreta       	     &   1925   &           8 hs  &   120 hs         \\ \hline
-  Geometría I               	     &   1935   &           6 hs  &    90 hs         \\ \hline
-  Cálculo II                         &   1928   &           8 hs  &   120 hs         \\ \hline
-  Álgebra Lineal I                   &   1933   &           8 hs  &   120 hs         \\ \hline
-  Taller de informática              &   1927   &           6 hs  &    90 hs         \\ \hline
-  Cálculo III                        &   1929   &           8 hs  &   120 hs         \\ \hline
-  Probabilidades                     &   1987   &           8 hs  &   120 hs         \\ \hline
-  Estructuras Algebraicas            &   1993   &           8 hs  &   120 hs         \\ \hline
-  Taller de Resolución de Problemas  &   1994   &           4 hs  &    60 hs         \\ \hline
-  Estadística                        &   1991   &           6 hs  &    90 hs         \\ \hline
-  Física                             &   1930   &           6 hs  &    90 hs         \\ \hline
-  \emph{Total de horas ciclo básico }& \multicolumn{3}{r|}{\emph{1260hs}}            \\ \hline
-\end{tabular}
+  Cálculo I                          &   1921   &           8 hs  &   112 hs         \\ \hline
+  Matemática Discreta       	     &   1925   &           8 hs  &   112 hs         \\ \hline
+  Geometría I               	     &   1935   &           6 hs  &    84 hs         \\ \hline
+  Cálculo II                         &   1928   &           8 hs  &   112 hs         \\ \hline
+  Álgebra Lineal I                   &   1933   &           8 hs  &   112 hs         \\ \hline
+  Taller de informática              &   1927   &           6 hs  &    84 hs         \\ \hline
+  Cálculo III                        &   1929   &           8 hs  &   112 hs         \\ \hline
+  Probabilidades                     &   1987   &           8 hs  &   112 hs         \\ \hline
+  Estructuras Algebraicas            &   1993   &           8 hs  &   112 hs         \\ \hline
+   Estadística                        &   1991   &           6 hs  &    90 hs         \\ \hline
+  Física                             &   1930   &           6 hs  &    84 hs         \\ \hline
+  \emph{Total de horas ciclo básico }& \multicolumn{3}{r|}{\emph{1136hs}}            \\ \hline
+\end{tabularx}
 \end{center}
 
 
-\paragraph{Ciclo Superior} Consta de 10 materias
+\paragraph{Ciclo Superior} Consta de 13 materias
 obligatorias.
 
 \begin{center}
-\begin{tabular}{|l|r|r|r|}\hline
+\begin{tabularx}{1\textwidth}{|>{\raggedleft\arraybackslash}X |
+>{\raggedleft\arraybackslash}X |
+>{\raggedleft\arraybackslash}X |
+>{\raggedleft\arraybackslash}X |}
+\hline
   \rowcolor[gray]{.9}
 \emph{Asignaturas del ciclo superior  }  & Código &\emph{Horas semanales} &  \emph{Horas totales}\\ \hline
-Inglés  (Anual)                            & 1976   &           4 hs        &     120 hs           \\ \hline
-Estudio de la Realidad Nacional          & 6235   &           2 hs        &      30 hs           \\ \hline
-Cálculo Numérico  Computacional          & 2030   &           8 hs        &     120 hs           \\ \hline
-Topología                                & 1917   &           9 hs        &     135 hs           \\ \hline
-Álgebra Lineal Aplicada                  & 2261   &           7 hs        &     105 hs           \\ \hline
-Variable Compleja y Análisis de Fourier  & 2262   &           9 hs        &     135 hs           \\ \hline
-Medida e Integración                     & 2263   &           8 hs        &     120 hs           \\ \hline
-Ecuaciones Diferenciales                 & 1913   &           8 hs        &     120 hs           \\ \hline
-Geometría Diferencial                    & 1915   &           8 hs        &     120 hs           \\ \hline
-Modelos Matemáticos                      & 2265   &           6 hs        &      90 hs           \\ \hline
-Modelos de regresión y metodos empíricos  & XXXX  &           6 hs 
-& 90 hs   \\ \hline
-\emph{Total de horas ciclo superior }    &\multicolumn{3}{r|}{\emph{1095 hs}}                    \\ \hline
-\end{tabular}
+Inglés  (Anual)                            & 1976   &           4        &     56           \\ \hline
+Estudio de la Realidad Nacional          & 6235   &           2        &      28           \\ \hline
+Cálculo Numérico  Computacional          & 2030   &           8        &     112           \\ \hline
+Topología                                & 1917   &           8        &     112           \\ \hline
+Álgebra Lineal Aplicada                  & 2261   &           8        &     112           \\ \hline
+Fundamentos de Análisis           &   xxxx &             8        &             112    \\ \hline
+Variables Complejas   & 1911   &           8        &     112           \\ \hline
+Medida e Integración                     & 2263   &           8        &     140           \\ \hline
+Ecuaciones Diferenciales                 & 1913   &           8        &     112           \\ \hline
+Geometría Diferencial                    & 1915   &           8        &     112           \\ \hline
+Modelos de regresión y metodos empíricos  & XXXX  &           6 
+& 90   \\ \hline
+Introducción a las Ecuaciones en Derivadas Parciales &   2212  & 8 & 112  \\ \hline
+Análisis Funcional               &    1916   &       8        &            112      \\ \hline
+\emph{Total de horas ciclo superior }    &\multicolumn{3}{r|}{\emph{1322 hs}}                    \\ \hline
+
+\end{tabularx}
 \end{center}
 
 
 
 \paragraph{Ciclo de Especialización} Dirigido a introducir al alumno
-en el estudio de una rama específica de la matemática a su
-elección. Incluye un \emph{seminario}
- destinado a orientar al alumno en su decisión.  Además consta de asignaturas
-\emph{optativas}, en un mínimo de dos y un máximo de cuatro que
-reúnan en total una suma de 270 horas y un \emph{trabajo final} de
-150 horas. Las optativas deberán ser realizadas bajo la dirección
-de un docente tutor del Departamento de Matemática a elección del
-alumno y con anuencia del Consejo Departamental y de la Comisión
-Curricular de la Licenciatura en Matemática. El tutor orientará al
-estudiante en la selección de las optativas, que podrán ser
-realizadas en este Departamento o en otro Departamento de la
-Universidad. El trabajo final deberá ser realizado bajo la
-dirección del docente tutor de acuerdo a las normativas de la
-Facultad de Ciencias Exactas Fco-Qcas y Naturales.
-
-
-Actualmente el Dpto. brinda las siguientes orientaciones.
+en la investigación en una rama específica de la matemática de su elección o a introducirlo en las aplicación de la matemática a un problema de modelización o aplicación de la matemática a un problema de origen tecnológico, social, productivo, etc. 
 
-
-
-
-\begin{enumerate}
-\item[A.] Análisis Matemático,
-\item[B.]  Matemática Aplicada,
-\item[C.] Didáctica de la Matemática,
-\item[D.] Estadística.
-\item[E.] Geometría, Álgebra y Grupos de Lie.
-\end{enumerate}
-
-Propuestas de otras orientaciones serán evaluadas por la C. C.
-permanente de la carrera.
+Consta de dos asignaturas optativas una electiva y contempla la realización de un trabajo final. La realización del trabajo final está regulado por la Res. CD. FCEFQyN N0 208/91 o aquella que la suplante.  \textcolor{red}{Deberíamos o ver en que estado está la modificación  de la 208/91 o  si debemos hacer un reglamento específico}.  
 
 \begin{center}
-\begin{tabular}{|l|r|r|}\hline
+\begin{tabularx}{1\textwidth}{|>{\raggedleft\arraybackslash}X |
+>{\raggedleft\arraybackslash}X |
+>{\raggedleft\arraybackslash}X |
+>{\raggedleft\arraybackslash}X |}
+\hline
   \rowcolor[gray]{.9}
-\emph{Asignaturas del ciclo de especialización  } & Código & \emph{Horas totales}      \\ \hline
-Seminario de Especialización                      & 2264   &   45 hs                   \\ \hline
-Optativas                                         &        &  270 hs                   \\ \hline
-Trabajo Final                                     & 2038   &  150 hs                   \\ \hline
-\emph{Total de horas ciclo de especialización}    & \multicolumn{2}{r|}{\emph{465 hs}} \\ \hline
-\end{tabular}
+\emph{Asignaturas del ciclo de especialización  } & Código&\emph{Horas semanales} & 
+\emph{Horas totales}      \\ \hline
+Optativa I                     &           &            8        &         112        \\ \hline
+ Electiva                 &            &    6                &        84          \\ \hline
+ Optativa II                    &            &           10         &          140        \\ \hline
+ Trabajo Final                     & 2265         &       10             &          140        \\ \hline
+\emph{Total de horas ciclo de especialización}    & \multicolumn{3}{r|}{\emph{476 hs}} \\ \hline
+\end{tabularx}
 \end{center}
 
 
-\newpage
-
-\end{colortext}
 
 \subsubsection{Listado total de asignaturas}
-\fontsize{8pt}{8pt}\selectfont  
+\fontsize{10pt}{10pt}\selectfont  
 \begin{center}
-\begin{xtabular}{|l|r|l|r|r|}\hline
+
+\begin{tabularx}{1\textwidth}{|>{\raggedleft\arraybackslash}X |
+>{\raggedleft\arraybackslash}X |
+>{\raggedleft\arraybackslash}X |
+>{\raggedleft\arraybackslash}X |
+>{\raggedleft\arraybackslash}X |}
+\hline
+\hline
 \rowcolor[gray]{.9}\multicolumn{5}{|c|}{\textbf{Primer año}}                                                        \\ \hline
 
 \emph{Cuat. }  &\emph{Código}  & \emph{Materia}                              &    hr. sem.       &  hr. Tot.        \\ \hline
@@ -663,16 +612,15 @@ V             & 2261          & Álgebra Lineal Aplicada                  &
 
 \multicolumn{3}{|l|}{\textbf{Total de Horas cuatrimestre V}}                &\textbf{22}           &\textbf{308}         \\ \hline
 VI             & 1930          & Física                                     &            6        &           84       \\ \hline
-VI             & 2262          & Variable Compleja                          &             8       &         112         \\ \hline
+VI             & 1911          & Variables Complejas                          &             8       &         112         \\ \hline
 VI             & 6235          & Estudio de la Realidad Nacional *          &               2     &         28         \\ \hline
 VI            & 1917          & Topología                                   &              8      &        112          \\ \hline
 \multicolumn{3}{|l|}{\textbf{Total de Horas cuatrimestre VI} }              &\textbf{24}           &\textbf{336}         \\ \hline
   
 \rowcolor[gray]{.9}\multicolumn{5}{|c|}{\textbf{Cuarto año}}                                                        \\ \hline
-VII           & 2263          & Medida e Integración                       &             8       &            112      \\ \hline
-
+VII           & 2263          & Medida e Integración                       &             10       &            140     \\ \hline
 VII            & 1913          & Ecuaciones Diferenciales                   &           8         &        112          \\ \hline
-\multicolumn{3}{|l|}{\textbf{Total de Horas cuatrimestre VII}}              & \textbf{24}          &\textbf{336}         \\ \hline
+\multicolumn{3}{|l|}{\textbf{Total de Horas cuatrimestre VII}}              & \textbf{18}          &\textbf{252}         \\ \hline
 VIII             & 1915          & Geometría Diferencial                      &              8     &        112          \\ \hline
 VIII           & xxx           & Modelos de regresión y metodos empíricos                  &              6      &         84         \\ \hline
 
@@ -680,25 +628,20 @@ VIII           & xxx           & Modelos de regresión y metodos empíricos
 
 VIII          & 2212           & Introducción a las Ecuaciones en Derivadas Parciales & 8  & 112  \\ \hline
 
-\multicolumn{3}{|l|}{\textbf{Total de Horas cuatrimestre VIII}}             & \textbf{20}          &\textbf{280}         \\ \hline
+\multicolumn{3}{|l|}{\textbf{Total de Horas cuatrimestre VIII}}             & \textbf{22}          &\textbf{308}         \\ \hline
 
 \rowcolor[gray]{.9}\multicolumn{5}{|c|}{\textbf{Quinto año}}                                                        \\ \hline
-IX            &  1916         &  Análisis Funcional               &           8         &            112      \\ \hline
-IX           &               & Optativa I                                &            10        &         140         \\ \hline
-\multicolumn{3}{|l|}{\textbf{Total de Horas cuatrimestre VII}}              & \textbf{}          &\textbf{}         \\ \hline
+IX            &  1916         &  Análisis Funcional               &           10         &            140     \\ \hline
+IX           &               & Optativa I                                &            8        &         112        \\ \hline
+X           &           & Electiva                             &    6                &        84          \\ \hline
+\multicolumn{3}{|l|}{\textbf{Total de Horas cuatrimestre VII}}              & \textbf{24}          &\textbf{336}         \\ \hline
 X           &               & Optativa II                                &           10         &          140        \\ \hline
-X           & 2265          & Trabajo Final                              &                    &                  \\ \hline
-X           &           & Electiva                             &                    &                  \\ \hline
-
-\multicolumn{3}{|l|}{\textbf{Total de Horas cuatrimestre VIII}}             & \textbf{}          &\textbf{}         \\ \hline
-\multicolumn{4}{|l|}{\textbf{Total de Horas del Plan de estudios}}                               &\textbf{}         \\ \hline
-
-
+X           & 2265          & Trabajo Final                              &       10             &          140        \\ \hline
 
 
-
-
-\end{xtabular}
+\multicolumn{3}{|l|}{\textbf{Total de Horas cuatrimestre VIII}}             & \textbf{20}          &\textbf{280}         \\ \hline
+\multicolumn{4}{|l|}{\textbf{Total de Horas del Plan de estudios}}                               &\textbf{}         \\ \hline
+\end{tabularx}
 \end{center}
 \normalsize
 
@@ -858,10 +801,10 @@ errores.
 Bibliografía sugerida: \cite{roederer, sears}.
 
 
-\item\textbf{Variable Compleja:}
-Funciones analíticas. Desarro\-llos en serie de potencias. Fórmula
+\item\textbf{Variables Complejas:}
+Funciones analíticas. Desarrollos en serie de potencias. Fórmula
 y teorema de Cauchy. Singularidades. Series de Laurent. Cálculo de
-residuos. Mapeo conforme. Aplicaciones a problemas de Dirichlet y Poisson.
+residuos. Teorema del módulo máximo. Mapeo conforme. Aplicaciones a problemas de Dirichlet y Poisson.
 
 Bibliografía sugerida: \cite{ahlfors, churchill,conway}
 
@@ -902,13 +845,12 @@ Conjuntos compactos y conexos. Espacios de funciones Teorema de Arzela-Ascoli. T
 Bibliografía Sugerida: \cite{dugundji, kelley,
 munkres, morris1989topology,  JohnMcCleary84,StefanWaldmann87,JohnB.Conway251, RobertAConover507}.
 
-\item\textbf{Ecuaciones Diferenciales (1913):} Ecuaciones de primer
-orden. Ecuaciones lineales de orden superior. Método de Frobenius. Sistemas lineales. Problemas de Sturm-Liouville.
+\item\textbf{Ecuaciones Diferenciales (1913):} Herramientas computacionales.  Ecuaciones lineales de orden $n$. Teoremas de separación y comparación de Sturm. Método de Frobenius. . Problemas de Sturm-Liouville. Funciones especiales de la Física-Matemática.  Sistemas lineales. 
 
 Bibliografía Sugerida: \cite{ GeorgeFinlaySimmons487,WilliamE.Boyce496, MorrisW.Hirsch540,JorgeSotomayor513,BarbaraD.MacCluer515,RichardS.Palais519,GarrettBirkhoff526}
 
 
-\item\textbf{Introducción a las Ecuaciones en Derivadas Parciales (2212):}  Ecuaciones de la Física-Matemática. Ecuaciones de primer orden. Método de características. Ecuación de Laplace. Teorema del valor medio. Teorema del máximo. Desigualdad de Harnack. Ecuación del Calor. Nucleo del calor. Operadores de convolución.  Método de Fourier. Problemas de Sturn-Liouville.  Ecuación de ondas. Método de características. Leyes de conservación. Formula de D'{}Alembert. Medias esféricas. 
+\item\textbf{Introducción a las Ecuaciones en Derivadas Parciales (2212):}  Ecuaciones de la Física-Matemática. Ecuaciones de primer orden. Método de características. Ecuación de Laplace. Teorema del valor medio. Teorema del máximo. Desigualdad de Harnack. Ecuación del Calor. Nucleo del calor. Operadores de convolución.  Método de Fourier.    Ecuación de ondas. Método de características. Leyes de conservación. Formula de D'{}Alembert. Medias esféricas. 
 
 Bibliografía sugerida: \cite{LawrenceC.Evans271,WalterCraig494,AlexanderKomech496,JulianFernandezBonder511,DavidBorthwick689,FritzJohn692,SandroSalsa693,AndrasVasy695,YehudaPinchover697,PavelDrabek698, AslakTveito699}.
 
@@ -919,7 +861,7 @@ Integral de
 Lebesgue. Lema de Fatou y Teorema de la Convergencia Mayorada.
 Teorema de Fubini. Espacios de Banach y de Hilbert. Espacios
 $L^p$. Espacio  $L^2$. Bases ortonormales. Series de Fourier.
-Funciones de variación acotadas y absolutamente continuas. 
+Funciones de variación acotadas y absolutamente continuas. Medidas abstractas. 
 
 Bibliografía sugerida: \cite{favazo, loeve, rudin, EliasM.Stein105,TerenceTao123,A.N.Kolmogorov682,DavidM.Bressoud121,wheeden2015measure}.
 
@@ -929,62 +871,67 @@ Bibliografía sugerida: \cite{favazo, loeve, rudin, EliasM.Stein105,TerenceTao12
 
  Técnicas de remuestreo. Métodos de regresión en alta dimensión. Ridge, Lasso, Elastic net. Grouped Lasso. Ensambles. Componentes principales para regresión. Selección de Variables. Modelos Aditivos. Métodos basados en Árboles. Árboles para regresión y clasificación. Datos Faltantes. Boosting
 
+ 
+ \item\textbf{Análisis Funcional(1916):}  Teoremas de Hahn-Banach. Principio de acotación uniforme.  Teoremas de la aplicación abierta y del grafo cerrado. Topologías débiles. Teoremas de Banach-Alaoglu,  Kakutani y  Eberlein–Šmulian. Espacios separables. Espacios de Hilbert. Bases ortonormales. Teorema de Lax-Milgram. Operadores en espacios de Hilbert. Operadores compactos. Descomposición espectral de operadores compactos autoadjuntos. Aplicaciones a problemas de Sturm-Liouville.
+
+Bibliografía sugerida: \cite{HaimBrezis130,EmmanueleDibenedetto159,FrancisClarke335,JohnBConway489,MichelWillem870,WalterRudin1039, MischaCotlar1454}.
 
+\item\textbf{Trabajo Final (2038):} El trabajo
+final se elaborará a partir de alguna/s de las siguientes
+alternativas:
+\begin{itemize}
+\item un trabajo de investigación realizado por el alumno,
+\item un trabajo de síntesis de artículos de investigación publicados en
+revistas de reconocido prestigio,
+\item una experiencia de aplicación de la matemática.
+\end{itemize}
 
+El alumno realizará una monografía de  que a posteriori defenderá
+oralmente siguiendo las normas establecidas por la Facultad de Ciencias Exactas, Fís-Qcas y Naturales.
 
 \end{enumerate}
 
-\subparagraph{Ciclo de Especialización} Consta de la siguiente
-asignatura obligatoria:
 
 
 
-\textbf{Seminario de Especialización (2264):} Espacio curricular
-destinado a orientar al alumno en torno a su decisión de la línea
-a seguir en sus estudios. Para su implementación, cada año el
-Consejo Departamental designará un docente encargado de coordinar
- un ciclo de charlas donde  integrantes de cada uno
-de los proyectos de investigación del Dpto. y, eventualmente,
-invitados externos al Dpto, informarán al estudiante sobre las
-distintas líneas de investigación que podría elegir para el ciclo
-de especialización. El alumno obtendrá la regularidad con el 80\%
-de asistencia y la aprobación con la elaboración de un escrito
-relacionado con el tema escogido, para lo cual previamente elegirá
-un docente tutor que asesorará la elaboración del escrito y
-evaluará el mismo junto al docente coordinador.
 
 
+\subsubsection{Transversalidad de contenidos y metodología: explicitación de los contenidos y metodologías transversales en los diferentes campos disciplinares o en espacios interdisciplinares.}
 
-\textbf{Plan de especialización:} Es el plan que deberán presentar el alumno con el  docente tutor al Consejo Departamental, con el detalle de las materias optativas que cursará el alumno y un proyecto de Trabajo Final. Es requisito para presentar el mismo tener
+Se consideran las siguientes temáticas, cuyo abordaje será transversal al plan de estudios. 
 
-\begin{itemize}
-\item aprobadas todas las asignaturas de los primeros dos años de la carrera.
-\item regularizadas el 70\% de las horas correspondientes a las asignaturas de tercer año.
-\item aprobado el \emph{Seminario de especialización}.
-\end{itemize}
-Habiendo recibido el plan, el Consejo Departamental lo analizará con el asesoramiento de la Comisión Curricular Permanente de la Licenciatura en Matemática, para luego elevarlo a la Secretaría Académica de la Facultad para su aprobación.
-\hyphenation{si-guien-tes}
 
-\textbf{Trabajo Final (2038):} El trabajo
-final se elaborará a partir de alguna/s de las siguientes
-alternativas:
-\begin{itemize}
-\item un trabajo de investigación realizado por el alumno,
-\item un trabajo de síntesis de artículos de investigación publicados en
-revistas de reconocido prestigio,
-\item una experiencia de aplicación de la matemática en grupos interdisciplinarios
-y/o en empresas.
-\end{itemize}
+\begin{description}
 
-El alumno realizará una monografía que a posteriori defenderá
-oralmente siguiendo las normas establecidas por la Facultad de Ciencias Exactas, Fís-Qcas y Naturales.
+\item{\textbf{Ciencia de datos}} \textcolor{green}{Definirla} Probabilidades (1987), Estadística (1991) y Modelos de regresión y Métodos Empíricos (XXXX). \textcolor{green}{Completar}
+
+
+\item{\textbf{Informática}}\textcolor{green}{Completar}
+
+
+\item{\textbf{Formación socio-política-cultural y pedagógica}}\textcolor{green}{Completar}
+
+
+
+\item{\textbf{Matemática Pura}} Entendemos por matemática pura aquellos estudios  matemáticos originados en problemas de la propia matemática. El nacimiento de lo que hoy denominamos  matemática pura fue un hito importante en  nuestra ciencia pues implicó la definición epistemológica que la ciencia matemática es independiente del universo sensible. Esto abrió multiples nuevas líneas de investigación que, eventualmente, terminaron por nutrir también a la matemática aplicada. 
+
+Como un aprendizaje significativo de nuestra ciencia implica que podamos reflexionar sobre la matemática en si  misma y analizar críticamente sus conclusiones y resultados, todas las asignaturas específicas  de la carrera aportan al trayecto de formación en matemática pura. Sin embargo algunas se caracterizan por hacerlo con más claridad y profundidad. Tal es el caso de Topología (1917), Medida e Integraciónc(2263), Variables Complejas (1911), Geometría Diferencial (1915), Estructuras Algebraicas (1993) y Análisis Funcional (1916). 
+
+ 
+\item{\textbf{Matemática Aplicada}} Entendemos por matemática aplicada aquellas teorías matemáticas destinadas a resolver problemas originados en  otras ciencias o que provengan del estudio  del  mundo sensible. 
+
+Como un objetivo central en la actualidad es la formación integral e interdisciplinaria del estudiante, es importante que la mayor cantidad de  espacios curriculares posibles muestren aplicaciones de las teorías trabajadas. Sin embargo hay áreas de la matemática cuyo  nacimiento
+vino fuertemente inspirado en las aplicaciones y consideramos una obligación que esto se vea reflejado en el plan de estudios. Las materias en esas condiciones son: Cálculo numérico y computacional (2030), Algebra Lineal Aplicada (2261), Variables Complejas (1911), Ecuaciones Diferenciales (1913), Introducción a las Ecuaciones en Derivadas Parciales (2212), Física (1930).
 
 
 
-\subsubsection{Transversalidad de contenidos y metodología: explicitación de los contenidos y metodologías transversales en los diferentes campos disciplinares o en espacios interdisciplinares.} \textcolor{green}{Nuevo}
+
+\end{description}
+
+
 
 \subsubsection{Correlatividades}
- \textcolor{green}{Hay que cambiar}
+ 
 \fontsize{9pt}{9pt}\selectfont
 \begin{center}
 \begin{tabularx}{\textwidth}{|l|l|l|p{3cm}|X|X|X|X|}\hline
@@ -1010,48 +957,44 @@ II   & 1  &  2030  &Cálculo Numérico Computacional &1928 \newline 1927 &1921 &
 
 II & A &   1976 &  Inglés &--&--&--&-- \\ \hline
 
-II & 2 &1987 & Probabilidades  &1928 &-&---&1928\\ \hline
+II & 2 &1987 & Probabilidades  &1929 &1928&--&1929\\ \hline
 
 II & 2& 1993 &Estructuras Algebraicas &1933&1925&--& 1933\\ \hline
 
-II & 2 & 2261 & Algebra Lineal Aplicada &--&1933&--&1933 \\ \hline
+III & 1 & 2261 & Algebra Lineal Aplicada &--&1933&--&1933 \\ \hline
 
 III & 1 & 1991 & Estadística &1987&--&--&1987 \\ \hline
 
-III &1 & 1930 &Física  &--&1929&--&1929 \newline
-\\ \hline
 
-III & 1 & 1917 &Topología &-- &1929&--&1929 \\ \hline
+III & 1 & XXXX &Fundamentos de Análisis &-- &1929&--&1929 \\ \hline
 
-III & 2 &2262 &Variable Compleja y \newline Análisis de Fourier &1917&-&---&1917 \\ \hline
+III &2 & 1930 &Física  &--&1929&--&1929 \newline
+\\ \hline
+III & 2 & 1917 &Topología &-- &XXXX&--&XXXX \\ \hline
 
-III & 2 &2263 &Medida e Integración &1917&--&--&1917 \\ \hline
+III & 2 &2262 &Variables Complejas  &XXXX&-&---&XXXX \\ \hline
 
-III & 2 &1994 &Taller de Resolución de Problemas  &--&1925\newline
-1935\newline
-1928 &--&1925 \newline
-1935\newline
-1928\\ \hline
 
-III & 2 & 2264 & Seminario de Especialización &1917\newline
-1993&--&--&1917\newline
-1993\\ \hline
 
 III & 2 & 6235 & Estudio de la Realidad Nacional &--&--&--&--\\ \hline
 
-
+IV & 1 &2263 &Medida e Integración &XXXX&--&--&XXXX \\ \hline
 
 IV & 1 & 1913 & Ecuaciones Diferenciales &1917\newline
 2261&--&--&1917\newline
 2261\\ \hline
 
-IV & 1 & 1915 & Geometría Diferencial & 1917\newline
-2261 &--&--& 1917\newline
+
+
+IV & 2 & 1915 & Geometría Diferencial & XXXX\newline
+2261 &--&--& XXXX\newline
 2261\\ \hline
 
-IV & 2 & 2265 & Modelos Matemáticos &1913&2261\newline
-2030&--&1913\newline
-2030\\ \hline
+IV & 2 & 2265 & Modelos de Regresión y Métodos Empíricos &1991&--&--&1991\\ \hline
+
+IV & 2 & 2212 &  Introducción a las Ecuaciones en Derivadas Parciales
+  &1913&--&--&1913\\ \hline
+
 
 IV & 2 & 2038 & Trabajo Final & 2261 &2263\newline 1976 &--& 2261\newline
 2263\newline 1976 \\ \hline
@@ -1070,20 +1013,22 @@ IV & 2 & 2038 & Trabajo Final & 2261 &2263\newline 1976 &--& 2261\newline
 
 \subsubsection{Otros requisitos necesarios para el cumplimiento del Plan de Estudios: señalar los seminarios, trabajos de campo, prácticas profesionales, residencias, idiomas, trabajos de tesis u otros requisitos exigidos para el otorgamiento del título.} 
 
-
+\textcolor{green}{Es la idea pedir horas en cursos extracurriculares, pasantías, en proyectos por ejemplo?}
 
 \subsubsection{Articulación con otros planes de estudio}
 
-\textcolor{green}{Hay que modificar}
+\textcolor{green}{Completar, con el profesorado por ejemplo}
 
-\paragraph{Equivalencias entre el plan nuevo y el plan 2001} \fontsize{11pt}{11pt}\selectfont
+\paragraph{Equivalencias entre el plan nuevo y el plan 2008} \fontsize{11pt}{11pt}\selectfont
 \begin{center}
-\begin{tabular}{|l|l|}\hline
+\begin{tabularx}{1\textwidth}{|>{\raggedright\arraybackslash}X |
+>{\raggedright\arraybackslash}X |}
+\hline
   \rowcolor[gray]{.9}
   \multicolumn{2}{|c|}{\textbf{Equivalencias}}\\\hline
 
-\emph{
-Plan 2001 versión 1}    & \emph{Plan 2008 Versión 2 }     \\
+   \emph{Plan 2008 Versión 2 }  &  \emph{
+Plan 2022 versión 0}   \\
 \hline
 
 
@@ -1111,30 +1056,39 @@ Probabilidades  (1987)      &Probabilidades    (1987)      \\ \hline
 
 Estructuras Algebraicas (1993)    & Estructuras Algebraicas (1993)  \\ \hline
 
-Inferencia Estadística  (2035)   &Estadística     (1991)        \\ \hline
+Estadística     (1991)    &Estadística     (1991)        \\ \hline
 
-Inferencia Estadística  (2035)   &Inferencia Estadística  (2035)       \\ \hline
 
 Física          (1930)                & Física    (1930)             \\ \hline
 
-Cálculo Avanzado         (1908)                 &Topología   (1917)       \\
+Topología   (1917) + coloquio & Fundamentos de Análisis   (XXXX)                      \\
+\hline
+Topología   (1917) + Variable Compleja y Análisis de Fourier (2262) & Fundamentos de Análisis   (XXXX)                      \\
+\hline
+
+
+
+Topología   (1917)                &Topología   (1917)       \\
 \hline
 
-Álgebra Lineal II (2061)  y coloquio      & Álgebra Lineal Aplicada (2261)       \\\hline
+Álgebra Lineal Aplicada (2261)      & Álgebra Lineal Aplicada (2261)       \\\hline
 
-Variables Complejas (1911) y coloquio                & Variable Compleja y Análisis de Fourier (2262) \\
+Variable Compleja y Análisis de Fourier (2262) & Variables Complejas (1911) + coloquio                 \\
 \hline
 
-Teoría de la medida (2036)     & Medida e Integración (2263) \\
+Medida e Integración (2263)   & Medida e Integración (2263) \\
 \hline
 
 Ecuaciones Diferenciales    (1913)     &Ecuaciones Diferenciales (1913)   \\
 \hline
 
-Geometría de curvas y superficies  (2037)  & Geometría Diferencial (1915)
+Geometría de curvas y superficies  (2037)  & Geometría de curvas y superficies  (2037)
 \\\hline
 
-\end{tabular}
+Introducción a las Ecuaciones en Derivadas Parciales (2212) & 
+Introducción a las Ecuaciones en Derivadas Parciales (2212)
+\\\hline
+\end{tabularx}
 \end{center}
 
 \normalfont
@@ -1142,65 +1096,188 @@ Geometría de curvas y superficies  (2037)  & Geometría Diferencial (1915)
 
 
 \subsubsection{Análisis de la congruencia interna de la carrera}
-
+\fontsize{10pt}{10pt}\selectfont 
+ 
 \begin{center}
 
-
-\begin{longtable}{|p{.2\textwidth}|p{.3\textwidth}|p{.5\textwidth}|}\hline
+\begin{tabularx}{1.0\textwidth}{|>{\raggedright\arraybackslash}X |
+>{\raggedright\arraybackslash}X |
+>{\raggedright\arraybackslash}X |
+}\hline
   \rowcolor[gray]{.9}
   Alcance del Título
   & Perfil del Título
   & Contenidos y Actividades \\ \hline
-  Participar en equipos interdisciplinarios realizando tareas de asesoramiento en temas específicos.
-  &1) Vincula las áreas de conocimientos teóricos-metodológicos con otras ciencias.
+Participar en equipos interdisciplinarios que utilicen la matemática.
+  & \emph{Conocimientos:} 
+  
+ 
+ Probabilidades.
+    
+   Estadística.
+   
+   Ciencia de Datos.
+   
+   Ecuaciones diferenciales.
+   
+   Algebra Lineal. 
+   
+   Cálculo Numérico. 
+   
+   Informática.
+ 
+  
+  \emph{Capacidades para:} 
+    
+ 
+   Formular problemas en lenguaje matemático.
+   
+    Analizar grandes conjuntos de datos
+    
+   Formular problemas en lenguaje matemático.
+   
+   Contribuir en la construcción de modelos matemáticos a partir de situaciones reales.
+   
+    Utilizar las herramientas computacionales de cálculo numérico y simbólico para plantear y resolver problemas.
+    
+   Comunicarse con otros profesionales no matemáticos.
+   
+   Trabajar en equipos interdisciplinarios.
+ 
+  
+  
+  &
 
-  2) Resuelve problemas interdisciplinarios, donde resolver un problema incluye su resolución numérica y computacional. Para esto se requiere conocimientos de algoritmos y de programación.
+  
+ Se prevé desarrollar aplicaciones de la matemática y modelización en muchas de las asignaturas del plan de estudio, como por ejemplo,  Física, Variables Complejas, Cálculo Numérico, Álgebra Lineal Aplicada, Estadística, Ecuaciones Diferenciales (ordinarias y parciales), Modelos de Regresión y Métodos Empíricos y además, eventualmente, en el ciclo de especialización.
+  
+  El aprendizaje en el uso de herramientas computacionales para resolver problemas se realizará en  Taller de Informática, Ecuaciones Diferenciales (ordinarias y parciales), Modelos de Regresión y Métodos Empírico.
+  
+  Se planifica desarrollar la capacidad de manejar grandes conjuntos de datos en Modelos de Regresión y Métodos Empíricos.
 
-  3)Modela matemáticamente diferente situaciones.
 
-  4)Comunica conocimientos matemáticos de manera oral y escrita.
+ La inclusión de una materia electiva pone al estudiante en situación de comunicarse con profesionales y pares de otras ciencias.
+ 
+  \\ \hline
+
+  
+  Realizar actividades de investigación en proyectos de matemática pura o aplicada.
+  & 
+  \emph{Conocimientos:}  Esencialmente todos los impartidos en la carrera.
+  
+  \emph{Capacidades para:}
+  
+   Aprender, actualizarse y trabajar de manera autónoma.
+   
+  Plantear y resolver problemas de matemática pura.
+  
+ Idear demostraciones.
 
-  5)Comprende la construcción social del conocimiento, valorando el trabajo en equipo.
-  &1)En muchas de las asignaturas del plan de estudio, como por ejemplo, Taller de Informática, Física, Cálculo Numérico, Álgebra Lineal Aplicada, Estadística, Ecuaciones Diferenciales, Modelos Matemáticos y además, eventualmente, en el ciclo de especialización, se contempla un espacio de aplicación de técnicas matemáticas a problemas en contextos interdisciplinarios y la aplicación de algoritmos computacionales a la solución de problemas.
+  Construir y desarrollar argumentaciones lógicas con
+una identificación clara de hipótesis y conclusiones.
 
-  2)La realización de trabajos grupales es llevada a cabo en distintas asignaturas, tales como   Modelos Matemáticos, Taller de Informática, Estadística, Taller de Resolución de Problemas, etc.
+Extraer de una situación los rasgos más
+relevantes.
 
-  3)La práctica de comunicar el conocimiento es trasversal a casi todos los espacios curriculares, sin embargo, se pone en juego de una manera más crítica durante el trabajo final que desarrolla el estudiante.
+Iniciar investigaciones matemáticas bajo orientación
+de experto.
 
-  4) La perspectiva histórico social de la construcción del conocimiento matemático es trabajada en diversos espacios curriculares obligatorios, como por ejemplo Estructuras Algebraicas y optativos como didáctica de la Matemática y Epistemología. \\ \hline
 
-  Realizar actividades de investigación en proyectos de matemática pura o aplicada.
-  & 1) Tiene la capacidad de idear demostraciones.
+ Contribuir en la construcción de modelos matemáticos a partir de situaciones reales.
+ 
+Utilizar las herramientas computacionales de cálculo
+numérico y simbólico para plantear y resolver problemas.
+
+Analizar grandes conjuntos de datos
+
+Expresarse correctamente utilizando el lenguaje de la matemática.
+
+  
+  
+  &1) El aprendizaje significativo de la matemática implica que el estudiante se enfrente con prácticas de investigación pensadas para su desarrollo en el aula y coherentes con  la etapa de construcción del conocimiento por parte del mismo. Por consiguiente casi todos los espacios curriculares presuponen un aprendizaje en la metodología de la investigación en matemática pura y consecuentemente aportan en la dirección de desarrollar muchos de los perfiles indicados.
+  
+  2)Más específicamente, el trabajo final de la carrera está destinado a la realización de un proceso de investigación.
+  
+  3) El Trayecto en matemática pura aportará a las capacidades y conocimientos vinculados con la investigación en matemática pura. 
+  
+  3) En cuanto a los perfiles más vinculados a la investigacióon interdisciplinaria y al análisis de datos se desarrollan en: trayectos curriculares en matemática aplicada y ciencia de datos.
+  
+  \\ \hline
+ Intervenir como peritos matemáticos en organismos públicos o pri-
+vados tales como, INDEC, empresas que realicen desarrollos tecnológicos, bancos, compañias de seguro, etc.
+
+  &   
+   \emph{Conocimientos:}
+   
+   Muchos conocimientos pueden ser potencialemnte útiles para este alcance. Por la incidencia que han adquirido en la actualidad se destacan aquellos vinculados con el análisis de datos y las tecnologías de la información. Estas temáticas son abordadas en los trayectos de Ciencias de Datos e Informática. Además es relevante una formación en métodos numéricos
+   
+   
+   \emph{Capacidades para:}
+
+
+ Valorar y respetar la diversidad y la multiculturalidad,
 
-  2) Tiene la capacidad de integrarse a proyectos de investigación.
+Contribuir en la construcción de modelos matemáticos a partir de
+situaciones reales,
 
-  3) Puede continuar con estudios de posgrado.
-  &1) El aprendizaje significativo de la matemática implica que el estudiante se enfrente con prácticas de investigación pensadas para su desarrollo en el aula y coherentes con  la etapa de construcción del conocimiento por parte del mismo. Por consiguiente casi todos los espacios curriculares presuponen un aprendizaje en la metodología de la investigación matemática.
+ Utilizar las herramientas computacionales de cálculo numérico y simbólico para plantear y resolver problemas,
+ 
+ Analizar grandes conjuntos de datos,
+
+ Comunicarse con otros profesionales no matemáticos,
 
-  2)Más específicamente, el trabajo final de la carrera está destinado a la realización de un proceso de investigación. \\ \hline
+  Leer, escribir y exponer documentos en inglés, ası́ como comunicar-
+se con otros especialistas,
 
-  Intervenir como peritos matemáticos en instituciones tales como empresas que realicen desarrollos tecnológicos, bancos, compañías de seguro, etc.
-  &  1)Resuelve problemas interdisciplinarios, donde resolver un problema incluye su resolución numérica y computacional. Para esto se requiere conocimientos de algoritmos y de programación.
+ Trabajar en equipos interdisciplinarios.
 
-  2)Modeliza matemáticamente diferente situaciones.
+  
 
-  3)Resuelve problemas que involucren técnicas matemáticas.
-  &1)En muchas de las asignaturas del plan de estudio, como por ejemplo Taller de Informática, Física, Cálculo Numérico, Álgebra Lineal Aplicada, Estadística, Ecuaciones Diferenciales Ordinarias, Modelos Matemáticos y además, eventualmente, en el ciclo de especialización, se contempla un espacio de aplicación de técnicas matemáticas a problemas en contextos interdisciplinarios y la aplicación de algoritmos computacionales a la solución de problemas. \\ \hline
+  
+  
+  
+  &  Trayectos  de formación socio-político-cultural (????),  en ciencia de datos e informática
+  
+  \\ \hline
 
   Acceder a carreras de posgrado.
   & Puede continuar con estudios de posgrado.
-  & Claramente este es un  alcance general de una carrera universitaria y no se pueden identificar contenidos o actividades específicas a él. Sin embargo, se resalta que la realización de un trabajo final facilita el desempeño del estudiante en futuras carreras de posgrado.   \\ \hline
+  & Claramente este es un  alcance general de una carrera universitaria y no se pueden identificar contenidos o actividades específicas a él. Sin embargo, se resalta que la realización de un trabajo final facilita el desempeño del estudiante en futuras carreras de posgrado.  
+  
+  Por otra parte la la profundidad de los temas abordados en el trayecto de matemática pura permiten la inserción del egresado en programas de posgrado en instituciones de reconocido prestigio con una alta exigencia de nivel académico.  \\ \hline
 
   Participar de los equipos docentes dirigidos a la enseñanza de la matemática en los niveles superiores de enseñanza.
-  &  1) Comunica conocimientos matemáticos de manera oral y escrita.
-
-  2) Comprende la construcción social del conocimiento, valorando el trabajo en equipo.                  &  1)La práctica de comunicar el conocimiento es trasversal a casi todos los espacios curriculares.
+  
+  & 
+     \emph{Conocimientos:}
+   
+   Pedagogía?
+   
+   Didáctica? 
+   
+   
+   
+   
+   
+   \emph{Capacidades para:}
+  
+   Actuar con responsabilidad social y compromiso ciudadano,
+   
+ Aprender, actualizarse y trabajar de manera autónoma,
+ 
 
-  2)En la Orientación C: Didáctica de la Matemática, se estudian de manera sistemática los procesos de enseñanza y aprendizaje de la enseñanza de la matemática.
+ Valorar y respetar la diversidad y la multiculturalidad,
+ 
+ 
+ Actuar en contextos educativos y planificar actividades de enseñanza.
 
-  3) En la asignatura Taller de Resolución de Problemas se propicia la discusión sobre el proceso de enseñanza-aprendizaje.\\ \hline
-\end{longtable}
+  
+  
+  &  Trayecto de formación socio-política-cultural y pedagógica.\\ \hline
+\end{tabularx}
 \end{center}
+\normalsize
+
 
 \subsubsection{Criterios para orientar la implementación del Plan de Estudio en coherencia con las propuestas epistemológicas y metodológicas que lo constituyen. Seguimiento y acompañamiento académico a la implementación, gestión y evaluación del Plan.
 } \textcolor{green}{Nuevo}
@@ -1210,9 +1287,12 @@ Geometría de curvas y superficies  (2037)  & Geometría Diferencial (1915)
 
 
 \subsection{Personal docente}
+
+\textcolor{red}{No lo modifiqué. Creo que cuando presentemos el nuevo plan de estudios estaremos en condiciones de citar nuestro plan de desarrollo.}
+
 El Departamento de Matemática cuenta con una planta docente
 capacitada para llevar a cabo la totalidad del Plan de
-Licenciatura propuesto en lo que \linebreak respecta a las
+Licenciatura propuesto en lo que  respecta a las
 asignaturas específicas de matemática. Además se solicitará apoyo
 al Dpto. de Física para el dictado de Física y a la facultad de
 Cs. Humanas para el dictado de Inglés y Estudio de la Realidad Nacional,
@@ -1227,6 +1307,8 @@ importante incrementar la planta docente hasta que cada docente
 pueda ocuparse del dictado de sólo una materia por cuatrimestre.
 
 \subsection{Personal técnico y administrativo}
+\textcolor{red}{No lo modifiqué. Creo que cuando presentemos el nuevo plan de estudios estaremos en condiciones de citar nuestro plan de desarrollo.}
+
 La Universidad, la Facultad de Ciencias Exactas, Físico-Químicas y
 Naturales y el Dpto. de Matemática cuentan con personal
 administrativo suficiente para desarrollar esta carrera.
@@ -1242,6 +1324,8 @@ especializado en informática.
 
 
 \section{Infraestructura Edilicia y Equipamiento}
+\textcolor{red}{No lo modifiqué. Creo que cuando presentemos el nuevo plan de estudios estaremos en condiciones de citar nuestro plan de desarrollo.}
+
  La Universidad
 Nacional de Río Cuarto dispone de las aulas, oficinas y equipos de
 computación necesarios para el desarrollo normal de esta carrera.
@@ -1256,7 +1340,7 @@ la Nación.
 
 \section{Asignación presupuestaria que demanda su
 implementación}
-
+\textcolor{red}{No lo modifiqué. Creo que cuando presentemos el nuevo plan de estudios estaremos en condiciones de citar nuestro plan de desarrollo.}
 La asignación presupuestaria necesaria para el desarrollo de esta
 carrera no se verá incrementada respecto del desarrollo de la
 Licenciatura vigente.
@@ -1264,12 +1348,13 @@ Licenciatura vigente.
 
 \section{Síntesis de la Propuesta presentada. }
 
+\textcolor{red}{Para el final}
 
 
 
 \bibliographystyle{apalike}
 \bibliography{Bibliografia/Analisis,Bibliografia/Algebra,%
-Bibliografia/Geometria,Bibliografia/BaseBibliografia,Bibliografia/Informatica,Bibliografia/Topologia,Bibliografia/TeoriaMedida,Bibliografia/probabilidad}
+Bibliografia/Geometria,Bibliografia/BaseBibliografia,Bibliografia/Informatica,Bibliografia/Topologia,Bibliografia/TeoriaMedida,Bibliografia/probabilidad,Bibliografia/funcional}
 
